\documentclass[a4paper,oneside,openany]{tufte-book}
\usepackage[utf8x]{inputenc}
\usepackage[german]{babel}
\usepackage{microtype} % Improves character and word spacing
\usepackage{booktabs} % Better horizontal rules in tables
\usepackage{ucs}
\usepackage{amsmath}
\usepackage{mathtools}
\usepackage{amsfonts}
\usepackage{amssymb}
\usepackage{graphicx}
\usepackage{color}
\usepackage{listings}
\usepackage{caption}
\usepackage{upgreek}

\makeatletter% since we're using commands with @ in their name

\let\origappendix\appendix% save a copy of the original meaning of \appendix
\renewcommand{\appendix}{%
  \origappendix% do all the original \appendix stuff
  \titlecontents{chapter}%
    [0em] % distance from left margin
    {\vspace{1.5\baselineskip}\begin{fullwidth}\LARGE\rmfamily\itshape} % above (global formatting of entry)
    {\hspace*{0em}\appendixname~\thecontentslabel: } % before w/label (label = ``2'')
    {\hspace*{0em}} % before w/o label
    {\rmfamily\upshape\qquad\thecontentspage} % filler + page (leaders and page num)
    [\end{fullwidth}] % after
  \titleformat{\chapter}%
    [display]% shape
    {\relax\ifthenelse{\NOT\boolean{@tufte@symmetric}}{\begin{fullwidth}}{}}% format applied to label+text
    {\itshape\huge Anhang~\thechapter}% label
    {0pt}% horizontal separation between label and title body
    {\huge\rmfamily\itshape}% before the title body
    [\ifthenelse{\NOT\boolean{@tufte@symmetric}}{\end{fullwidth}}{}]% after the title body
  \setcounter{secnumdepth}{0}% ``number'' the appendices
  \renewcommand{\thefigure}{\@arabic\c@figure}% define \thefigure to use only the figure number (1), not A.1
  \renewcommand{\thetable}{\@arabic\c@table}%
  %
  % Add any other special appendix-related code here.
  %
}
\makeatother% restore the special meaning of @

\author{Schett Matthias}
\title{SEN-\"{U}bung 2.7}

\begin{document}

\definecolor{gray}{rgb}{0.9,0.9,0.9} % background color for the listings

\lstset{language=[Visual]Basic, morekeywords={param, local}, breaklines=true, tabsize=4, showstringspaces=false,backgroundcolor=\color{gray}, numbers=left} % standard listing settings

\frontmatter

\maketitle
\tableofcontents
\mainmatter

\chapter{Aufgabe 1}

\section{L\"{o}sungsidee}

Es soll die Wetter Stations Verwaltung aus Übung 4 mithilfe der std::list neu implmentiert werden. Daher wird die Wetterstations Klasse unverändert übernommen.
In der WeatherStations\sidenote{Zuständig für die Verwaltung von Wetterstationen} Klasse werden jedoch einige Veränderungen durchgeführt.
Es gibt nun zum Beispiel nur mehr eine Membervariable, mWeatherStations vom Type std::list<WeatherStation>.
Durch die Verwedung von std::list\sidenote{std::list ist die Standardimplementierung einer doppelt verketteten List} ergeben sich auch für die Implementierung einige Unterschiede.
Weiters gilt es anzumerken, dass durch die Verwendung von std::list der Default Konstruktor, Default Copy Konstruktor, Default Destruktor und Default Assignement Operator 
verwendet werden kann, da Beispielsweise die Konstruktoren, die jeweiligen Konstruktoren der Membervariablen aufrufen.

\begin{fullwidth} \begin{lstlisting}[caption=Alte Implementierung Add]
bool WeatherStations::Add(WeatherStation const &ws){
    if(mNumberOfStations < mMaxNumber){
        mStationArray[mNumberOfStations] = ws;
        ++mNumberOfStations;
        return true;
    }
    return false;
}
\end{lstlisting} \end{fullwidth}

\begin{fullwidth} \begin{lstlisting}[caption=Neue Implementierung Add]
void WeatherStations::Add(WeatherStation const &ws){
    if(mWeatherStations.empty()){
        mWeatherStations.push_front(ws);
    } else{
        // using standard functions just push it to the back then sort and remove duplicates
        mWeatherStations.remove_if(RemoveWithName(ws));
        mWeatherStations.push_back(ws);
        mWeatherStations.sort(sortByName);
    }
}
\end{lstlisting} \end{fullwidth}

Auch die 2 Hilfsmethoden searchForColdest und searchForWarmest wurden neu implementiert

\begin{fullwidth} \begin{lstlisting}[caption=Implementierung der searchForColdest-Methode]
WeatherStation WeatherStations::searchForColdest() const{
    if(!mWeatherStations.empty()){
        std::list<WeatherStation>::const_iterator it = std::min_element(mWeatherStations.begin(), mWeatherStations.end(), findColdest);
        return *it;
    } else {
        throw WeatherException("No stations are defined");
    }
}
\end{lstlisting} \end{fullwidth}

Zu guter Letzt musste auch die PrintAll-Methode angepasst werden:

\begin{fullwidth} \begin{lstlisting}[caption=Implementierung PrintAll]
void WeatherStations::PrintAll(std::ostream &out) const{
    if(!mWeatherStations.empty()){
        for (std::list<WeatherStation>::const_iterator it = mWeatherStations.begin(); it != mWeatherStations.end(); ++it) {
            it->Print(out);
        }
    } else{
        out << "The list is empty" << std::endl;
    }
}
\end{lstlisting} \end{fullwidth}

Der Code findet sich ab Anhang \nameref{chap:Auf1}.\sidenote{Der Code der WeatherStation Klasse wird da diese nicht verändert wurde nicht mit angehängt.}

\section{Testf\"{a}lle}

Der Testreiber testet die Funktionnalität indem er zuerst ein Objekt erstellt, vier Wetterstationen versucht hinzuzfügen und anschließend sämtliche Methoden aufruft.
\paragraph{Ausgabe}
\begin{fullwidth}
\lstinputlisting[caption=Testfall Ausgabe]{WeatherStationManagerListVersion/OutputA1.txt}
\end{fullwidth}
\chapter{Aufgabe 2}

\section{L\"{o}sungsidee}

Es sollen verschiedene STL Algorithmen getestet werden. Dazu wird ein std::vector angeleget, der die in der Aufgabenstellung beschriebene Klasse Element aufnimmt.
Dieser std::vector wird mittels des Zufallszahlengenerators befüllt. Anschließend wird mittles des Copy Konstruktors eine Kopie erstellt und mittels des std::copy Algorithmus und eines
std::back\_inserters eine std::list als Kopie erstellt. Diese 3 Container werden unterschiedlich sortiert, das Original mit std::sort, der Kopierte std::vector mit std::stable\_sort und die Liste mit std::list.sort.

Über das Resultat lässt sich sagen, dass std::stable\_sort und std::list.sort beide das gleiche Ergebnis liefern, da beide eine Liste die nach den Zufalsswerten sortiert ist und gleichzeitig nach dem Index liefern.
Die Begürndung für dieses Verhalten findet sich in der offizielen Dokumentation\sidenote{Sorts the elements in the range (first,last) into ascending order, like sort, but stable\_sort preserves the relative order of the elements with equivalent values.(Quelle: \url{http://www.cplusplus.com/reference/algorithm/stable_sort/} Datum: 18-05-2013)}.

Der letzte Test erfolgt mit equal\_range der einen Bereich mit den gleichen Werte zurückgibt, aus welchem sich dann errechnen lässt, wie oft eine bestimmte Zufallszahl vorkommt.
Der Code findet sich ab Anhang \nameref{chap:Auf1}.

\section{Testf\"{a}lle}
\begin{fullwidth}
\lstinputlisting[caption=Testfall Ausgabe]{OutputA2.txt}
\end{fullwidth}
\backmatter

\appendix

\setboolean{@mainmatter}{true}
\chapter{Aufgabe 1}\label{chap:Auf1}
\begin{fullwidth}
\lstinputlisting[caption=Header der Wetterstationsverwaltung]{WeatherStationManagerListVersion/WeatherStations.h}
\lstinputlisting[caption=Implementierung der Wetterstationsverwaltung]{WeatherStationManagerListVersion/WeatherStations.cpp}
\lstinputlisting[caption=Testtreiber]{WeatherStationManagerListVersion/Main.cpp}
\end{fullwidth}
\chapter{Aufgabe 2}\label{chap:Auf2}
\begin{fullwidth}
\lstinputlisting[caption=Header der Elementklasse]{STLTest/Element.h}
\lstinputlisting[caption=Implementierung der ElementKlasse]{STLTest/Element.cpp}
\lstinputlisting[caption=Header des STL Tests]{STLTest/StlTest.h}
\lstinputlisting[caption=Implementierung des STL Tests]{STLTest/StlTest.cpp}
\lstinputlisting[caption=Testtreiber]{STLTest/Main.cpp}
\end{fullwidth}


\end{document}